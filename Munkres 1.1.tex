\documentclass{article}
\usepackage[a4paper, margin=.5in]{geometry}
\usepackage{amssymb}

\title{Submission 3.1}
\date{}
\begin{document}
\maketitle


\begin{enumerate}
      \item Check the distributive laws for $\cup$ and $\cap$ and DeMorgan's Laws.
            \begin{enumerate}
                  \item $A \cap (B \cup C) = (A \cap B) \cup (A \cap C)$\\
                        Let $x \in A \cap (B \cup C)$. It can then be said that $x \in A \land (x \in B \lor x \in C)$. Therefore, $(x \in A \land x \in B) \lor (x \in A \land x \in C)$. So, $(x \in A \cap B) \lor (x \in A \cap C)$, so $x \in (A \cap B) \cup (A \cap C)$. It can thus be said that $A \cap (B \cup C) \subset (A \cap B) \cup (A \cap C)$, and because all the steps are reversible, $A \cap (B \cup C) \supset (A \cap B) \cup (A \cap C)$. Therefore, $A \cap (B \cup C) = (A \cap B) \cup (A \cap C)$. (The same reasoning applies for (b) - (c) as well.)
                  \item $A \cup (B \cap C) = (A \cup B) \cap (A \cup C)$\\
                        Let $x \in A \cup (B \cap C)$. It can then be said that $x \in A \lor (x \in B \land x \in C)$. Therefore, $(x \in A \lor x \in B) \land (x \in A \lor x \in B)$. So, $x \in (A \cup B) \lor x \in (A \cup C)$, so $x \in (A \cup B) \cap (A \cup C)$.
                  \item $A - (B \cup C) = (A - B) \cap (A - C)$\\
                        Let $x \in (A - (B \cup C))$. It can then be said that $x \in A \land x \notin B \land x \notin C$. Thus, $(x \in A \land x \notin B) \land (x \in A \land x \notin C)$. Therefore, $x \in A - B \land x \in A - C$, so $x \in (A - B) \cap (A - C)$.
                  \item $A - (B \cap C) = (A - B) \cup (A - C)$\\
                        Let $x \in A - (B \cap C)$. It can then be said that $x \in A \land (x \notin B \lor x \notin C)$. Therefore, $(x \in A \land x \notin B) \lor (x \in A \land x \notin B)$. So $x \in (A - B) \lor x \in (A - C)$, so $x \in (A - B) \cup (A - C)$.
            \end{enumerate}
      \item Determine which of the following statements are true for all sets A, B, C, and D. If a double implication fails, determine whether one or the other of the possible implications holds. If an equality fails, determine whether the statement becomes true if the “equals” symbol is replaced by one or the other of the inclusions symbols $\subset$ or $\supset$.
            \begin{enumerate}
                  \item $(A \subset B$ and $A \subset C) \Leftrightarrow A \subset (B \cup C)$ is not true. However, $(A \subset B$ and $A \subset C) \Rightarrow A \subset (B \cup C)$ is.\\
                  $A \subset B$ and $B \subset B \cup C$. Because $A$ is a subset of something fully contained in $B \cup C$, it must also be a subset of $B \cup C$, so the forward implication works. Let $A = \{0, 1\}$, $B = \{0\}$, and $C = \{1\}$. Here, $A \subset B \cup C$, but $A \not\subset B$ and $A \not\subset C$, so the backwards implication fails.
                        \setcounter{enumii}{4}
                  \item $A - (A - B) = B$ is not true, but $A - (A - B) \subset B$ is.\\
                  $A - B$ = $\{x \mid x \in A \land x \notin B\}$, so $A - (A - B)$ = $\{x \mid x \in A \land \neg (x \in A \land x \notin B)\}$, so $\{x \mid x \in A \land (x \notin A \lor x \in B)\}$. This means $\{x \mid (x \in A \land x \notin A) \lor (x \in A \land x \in B)\}$. $(x \in A \land x \notin A)$ is a contradiction, so the statement simplifies to $\{x \mid x \in A \land x \in B\}$. This means everything within $A - (A - B)$ is contained within $B$, so $A - (A - B) \subset B$ is true, but if some element is contained within $B$ but not $A$, it will not be contained in $A - (A - B)$, so the equality fails.
                        \setcounter{enumii}{8}
                  \item $(A \cap B) \cup (A - B) = A$ is true.\\
                  $(A \cap B) \cup (A - B) = \{x \mid (x \in A \land x \in B) \lor (x \in A \land x \notin B)\}$. Therefore, $\{x \mid x \in A \land (x \in B \lor x \notin B)\}$. ($x \in B \lor x \notin B)$ is a tautology, so the statement simplifies to $\{x \mid x \in A\}$, which is the definition for the set $A$. All steps are reversible, so the equality holds.
                        \setcounter{enumii}{14}
                  \item $A \times (B-C) = (A \times B) - (A \times C)$ is true.\\
                  $A \times (B-C) = \{(a, b) \mid a \in A \land (b \in B \land b \notin C)\}$\\
                  $(A \times B) - (A \times C) = \{(a, b) \mid (a \in A \land b \in B) \land \neg (a \in A \land b \in C)\} = \{(a, b) \mid (a \in A \land b \in B) \land (a \notin A \lor b \notin C)\} = \{(a, b) \mid (a \in A \land b \in B \land a \notin A) \lor (a \in A \land b \in B \land b \notin C)\}$. The first part of the main or contains a contradiction, so the statement simplifies to $\{(a, b) \mid a \in A \land b \in B \land b \notin C\}$. All steps are reversible (I really hope), so the equality holds.
            \end{enumerate}
      \item
            \begin{enumerate}
                  \item Write the contrapositive and converse of the following statement: “If $x < 0$, then $x^{2} - x > 0$,” and determine which (if any) of the three statements are true.\\
                        The statement is true. $x^{2}$ is always positive, and if $x < 0$, then subtracting x will be equivalent to adding a positive number, so $x^{2} - x$ will be greater than 0.\\
                        The contrapositive of the statement is “If $x^{2} - x \leq 0$, then $x \geq 0$" and is also true because it is logically equivalent.\\ 
                        The converse is “If $x^{2} - x > 0$, then $x < 0$,” which is false. If x is 2, $x^{2} - x$ is 2, fulfilling the antecedent but not the consequent.
                  \item Do the same for the statement “If $x > 0$, then $x^{2} - x > 0$.”\\
                        The statement is false. If x is 1, $x^{2} - x$ is 0, so something true implies something false.\\
                        Therefore, the contrapositive, which is “If $x^{2} - x \leq 0$, then $x \leq 0$" is also false.\\
                        The converse, “If $x^{2} - x > 0$, then $x > 0$" is also false. If x is -1, the antecedent is fulfilled but the consequent is not.
            \end{enumerate}
      \item Let A and B be sets of real numbers. Write the negation of each of the following statements:
            \begin{enumerate}
                  \item For every $a \in A$, it is true that $a^{2} \in B$.\\
                        For at least one $a \in A$, it is true that $a^{2} \notin B$.
                  \item For at least one $a \in A$, it is true that $a^{2} \in B$.\\
                        For every $a \in A$, it is true that $a^{2} \notin B$.
                  \item For every $a \in A$, it is true that $a^{2} \notin B$.\\
                        For at least one $a \in A$, it is true that $a^{2} \in B$.
                  \item For every $a \notin A$, it is true that $a^{2} \in B$.\\
                        For at least one $a \notin A$, it is true that $a^{2} \notin B$.
            \end{enumerate}
      \item Let $\mathcal{A}$ be a nonempty collection of sets. Determine the truth of each of the following statements and their converses.
            \begin{enumerate}
                  \item $x \in \bigcup_{A \in \mathcal{A}} A \Rightarrow x \in A$ for at least one $A \in \mathcal{A}$.\\
                        This is true by the definition of an arbitrary union. The converse, “$x \in A$ for at least one $A \in \mathcal{A} \Rightarrow x \in \bigcup_{A \in \mathcal{A}} A$," is also true, again by definition.
                  \item $x \in \bigcup_{A \in \mathcal{A}} A \Rightarrow x \in A$ for every $A \in \mathcal{A}$.\\
                        This is false. A counterexample is given by $x = 0, A = \{0\}, B = \{1\}, \mathcal{A} = \{A, B\}$.\\ 
                        However, the converse, “$x \in A$ for every $A \in \mathcal{A} \Rightarrow x \in \bigcup_{A \in \mathcal{A}} A$," is true. If $x \in A$ for every $A \in \mathcal{A}$, then $x \in A$ for at least one $A \in \mathcal{A}$, which by definition implies that $x \in \bigcup_{A \in \mathcal{A}} A$.
                  \item $x \in \bigcap_{A \in \mathcal{A}} A \Rightarrow x \in A$ for at least one $A \in \mathcal{A}$.\\
                        This is true. By definition, $x \in \bigcap_{A \in \mathcal{A}} A$ implies that $x \in A$ for every $A \in \mathcal{A}$, which in turn implies $x \in A$ for at least one $A \in \mathcal{A}$.\\
                        However, the converse, “$x \in A$ for at least one $A \in \mathcal{A} \Rightarrow x \in \bigcap_{A \in \mathcal{A}} A$," is false. A counterexample is given by $x = 0, A = \{0, 1\}, B = \{1, 2\}, \mathcal{A} = \{A, B\}$. 
                  \item $x \in \bigcap_{A \in \mathcal{A}} A \Rightarrow x \in A$ for every $A \in \mathcal{A}$.\\
                        This is true by the definition of an arbitrary intersection. The converse, “$x \in A$ for every $A \in \mathcal{A} \Rightarrow x \in \bigcap_{A \in \mathcal{A}} A$," is also true by definition.
            \end{enumerate}
      \item Write the contrapositive of each statement in exercise 5.
            \begin{enumerate}
                  \item $x \notin A$ for every $A \in \mathcal{A} \Rightarrow x \notin \bigcup_{A \in \mathcal{A}} A$
                  \item $x \notin A$ for at least one $A \in \mathcal{A} \Rightarrow x \notin \bigcup_{A \in \mathcal{A}} A$
                  \item $x \notin A$ for every $A \in \mathcal{A} \Rightarrow x \notin \bigcap_{A \in \mathcal{A}} A$
                  \item $x \notin A$ for at least one $A \in \mathcal{A} \Rightarrow x \notin \bigcap_{A \in \mathcal{A}} A$
            \end{enumerate}
      \item Given sets A, B, and C, express each of the following sets in terms of A, B, and C, using the symbols $\cup$, $\cap$, and $-$.
            \begin{enumerate}
                  \item $D = \{x \mid x \in A$ and $(x \in B$ or $x \in C)\} = A \cap (B \cup C)$
                  \item $E = \{x \mid (x \in A$ and $x \in B)$ or $x \in C\} = (A \cap B) \cup C$
                  \item $F = \{x \mid x \in A$ and $(x \in B \Rightarrow x \in C)\} = A - (B - C)$\\
                  Let $x \in F$. It can then be said that $x \in A \land (x \in B \to x \in C)$, so $x \in A \land (x \notin B \lor x \in C)$, so $x \in A \land \neg(x \in B \land x \notin C)$. Therefore, $x \in A \land x \notin (B - C)$.
            \end{enumerate}
      \item If a set $A$ has two elements, show that $\mathcal{P}(A)$ has 4 elements. How many elements does $\mathcal{P}(A)$ have if $A$ has one element? Three elements? No elements? Why is $\mathcal{P}(A)$ called the power set of $A$?\\
            Let $A = \{a, b\}$, a set with 2 elements. $\mathcal{P}(A) = \{\varnothing, \{a\}, \{b\}, \{a, b\}\}$, a set with 4 elements.\\
            Let $A = \{a\}$, a set with 1 element. $\mathcal{P}(A) = \{\varnothing, \{a\}\}$, a set with 2 elements.\\
            Let $A = \{a, b, c\}$, a set with 3 elements. $\mathcal{P}(A) = \{\varnothing, \{a\}, \{b\}, \{c\}, \{a, b\}, \{a, c\}, \{b, c\}, \{a, b, c\}\}$, a set with 8 elements.\\
            Let $A = \varnothing$, a set with no elements. $\mathcal{P}(A) = \{\varnothing\}$, a set with 1 element.\\
            $\mathcal{P}(A)$ is called the power set of $A$ because where $A$ has $n$ elements, $\mathcal{P}(A)$ has $2^{n}$ elements. For each set in $\mathcal{P}(A)$, each element of $A$ can be either included or excluded. This results in choosing between one of two options $n$ times, resulting in $2^{n}$ possible options.
      \item Formulate and prove DeMorgan's laws for arbitrary unions and intersections.\\
            $A - \bigcup_{B \in \mathcal{B}} B = \bigcap_{B \in \mathcal{B}}(A - B)$\\
            Let $x \in A - \bigcup_{B \in \mathcal{B}} B$. It can then be said that $x \in A \land \neg (x \in B$ for at least one $B \in \mathcal{B})$. Therefore, $x \in A \land x \notin B$ for every $B \in \mathcal{B}$. So, $x \in (A - B)$ for every $B \in \mathcal{B}$, so $x \in \bigcap_{B \in \mathcal{B}}(A - B)$. Again, all the steps are reversible, showing equality. \\
            $A - \bigcap_{B \in \mathcal{B}} B = A - \bigcup_{B \in \mathcal{B}} B$\\
            Let $x \in A - \bigcap_{B \in \mathcal{B}}$. It can then be said that $x \in A \land \neg(x \in B$ for all $B \in \mathcal{B})$. Therefore, $x \in A \land x \notin B$ for at least one $B \in \mathcal{B}$. So, $x \in A \land x \notin \bigcup_{B \in \mathcal{B}} B$. Therefore, $x \in A - \bigcup_{B \in \mathcal{B}} B$. These steps are reversible, showing equality.
      \item Let $\mathbb{R}$ denote the set of real numbers. For each of the following subsets of $\mathbb{R} \times \mathbb{R}$ determine whether it is equal to the Cartesian product of two subsets of R.
      \begin{enumerate}
            \item \{$(x, y) \mid x$ is an integer\} is the Cartesian product $\mathbb{Z} \times \mathbb{R}$.
            \item \{$(x, y) \mid 0 < y \leq 1$\} is the Cartesian product $\mathbb{R} \times (0, 1]$.
            \item $\{(x, y) \mid y > x\}$ is not a Cartesian product.\\
            (0, 1) and (1, 2) are contained within the given subset. If the subset were a Cartesian product of two subsets of $\mathbb{R}$, (1, 1) and (0, 2) would then also have to be included in the subset. However, (1, 1) is not, so the subset must not be a Cartesian product.
            \item \{$(x, y) \mid $ x is not an integer and y is an integer\} is the Cartesian product $(\mathbb{R} - \mathbb{Z}) \times \mathbb{Z}$.
            \item $\{(x, y) \mid x^{2} + y^{2} < 1\}$ is not a Cartesian product.\\
            (0, 0.75) and (0.75, 0) are both contained within the given subset. If the subset were a Cartesian product of two subsets of $\mathbb{R}$, it would contain (0, 0) and (0.75, 0.75). However, (0.75, 0.75) is not contained, so the subset must not be a Cartesian product.
      \end{enumerate}
\end{enumerate}
\end{document}